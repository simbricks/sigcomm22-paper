\section{Introduction}
Our community expects research ideas to be implemented and evaluated as part of
a complete system ``end-to-end'' in realistic conditions.
%
End-to-end evaluation is important as many factors in each system component
affect the overall behavior in subtle and unpredictable ways.


Yet evaluation in full physical testbeds is frequently infeasible.
%
Work might require cutting edge commercial hardware that is not yet
available at the time of
publication~\cite{sharma:flexswitch,liu:incbricks,li:nopaxos,eris},
develop hardware extensions to existing proprietary
hardware~\cite{sharma:afq}, or propose entirely new ASIC hardware
architectures~\cite{bosshart:rmt,chole:drmt,sivaraman:packet_transactions,
sutherland:nebula,kaufmann:flexnic,jouppi:tpu,magaki:asic_clouds,
lin:panic}.
%
The trend towards increasingly specialized hardware, including
SmartNICs, programmable switches, and other accelerators, further
exacerbates this.
%
Finally, work on network protocols and congestion control necessitates
evaluation in large scale networks with hundreds to thousands of hosts.


% KEY POINT: simulation could help, but existing simulators fall short
When a full evaluation in a physical testbed is not possible,
simulation has long offered an alternative.
%
In networking, we use ns-2~\cite{software:ns2},
ns-3~\cite{software:ns3}, and OMNeT++~\cite{varga:omnetpp} to evaluate
protocols and algorithms;
computer architects rely on system simulators such as
gem5~\cite{binkert:gem5},
while hardware designers employ RTL simulators such as
Modelsim~\cite{software:modelsim} or
Verilator~\cite{software:verilator}.
%
While network systems do benefit from these
simulators~\cite{arashloo:tonic,mittal:irn,kaufmann:tas}, they do not
enable end-to-end evaluation, as no existing simulator simulates all
required components in a testbed: hosts, devices, and the full network.

In this paper, we demonstrate how to enable end-to-end network system simulation
by combining different simulators to cover the necessary functionality.
%
Instead of building a new simulator, throwing away decades of work, we connect
existing and new simulators -- for hosts, hardware devices, and networks -- into
full system simulations capable of running unmodified operating systems,
drivers, and applications.
%
Existing simulators, however, are standalone and not designed to be
combined with other simulators.
%
To achieve modular end-to-end simulation, we thus need to overcome
three technical challenges:
1) no interfaces to connect simulators together,
2) efficient, scalable, and correct synchronization of simulator clocks, and
3) combining mutually incompatible simulation models.


We present the design and implementation of \emph{\sysname, a modular
simulation framework for end-to-end network system simulations}.
%
\sysname defines interfaces for interconnecting simulators based on
natural component boundaries in physical systems, specifically
PCIe and Ethernet links.
%
Individual component simulators run in \emph{parallel} as separate processes,
and communicate via message passing only between connected peers through
optimized shared memory queues.
%
With this message transport, we co-design a protocol that leverages
simulation topology and latency at component boundaries for
\emph{efficient and accurate synchronization} of simulator clocks.
%
For scaling out simulations across physical hosts, we introduce
a proxy to forward messages over TCP or RDMA.

Currently, \sysname integrates QEMU~\cite{software:qemu} and
gem5~\cite{binkert:gem5} as host simulators,
Verilator~\cite{software:verilator} as an RTL hardware simulator for
hardware devices,
and ns-3~\cite{software:ns3}, OMNeT++~\cite{varga:omnetpp}, as well
as the Intel Tofino simulator~\cite{software:p4studio} for
network simulation.
%
Further, we have integrated open source RTL designs for the Corundum
FPGA NIC~\cite{forencich:corundum} and the
Menshen switch pipeline~\cite{wang:menshen} to showcase \sysname{}'s
generality.
%
We have also implemented fast behavioral simulators, \eg for
the Intel X710 40G NIC~\cite{spec:intel_x710}, and ported the
FEMU NVMe SSD model~\cite{li:femu} into \sysname.
%
In combination, these simulators enable a broad range of end-to-end
configurations for different use-cases.

In our evaluation, we demonstrate that \sysname enables end-to-end simulation of
existing network systems at small and large scales.
%
We also reproduce key results from congestion
control~\cite{alizadeh:dctcp},
in-network compute~\cite{li:nopaxos},
and FPGA NIC design~\cite{forencich:corundum} in \sysname.
%
\sysname obtains more realistic results compared to ns-3 in isolation
(\S\ref{ssec:bg:sysresearch}).
%
\sysname also scales to 1000 hosts and NICs with only a
14\% increase in simulation time compared to a 40-host simulation
(\S\ref{ssec:eval:scalability}).
%
Finally, \sysname provides deep visibility and control of low-level system
behaviors, facilitating evaluation and performance debugging
(\S\ref{ssec:eval:corundum}).

\smallskip\noindent
We make the following technical contributions:
\begin{itemize}
  \item \emph{Modular architecture for end-to-end system simulation}
    (\S\ref{ssec:design:interface}) combining host, device, and network
    simulators.

  \item \emph{Co-designed message transport and synchronization
    mechanism for parallel and distributed simulations}
    (\S\ref{ssec:design:syncproto}, \S\ref{ssec:design:transport}) leveraging
    pairwise message passing to efficiently ensure correct simulation, even
    at scale.

  \item \sysname, a \emph{prototype implementation} of our
    architecture (\S\ref{sec:impl}) with integrations for
    existing and new simulators.
    %
\end{itemize}

\noindent
\sysname is available open source at \url{https://simbricks.github.io}

\noindent
This work does not raise any ethical issues.

