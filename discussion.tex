\section{Looking Forward}
\label{sec:discussion}

\paragraph{Validation.}
To obtain representative simulation results, users have to
validate simulators and configuration parameters against physical
testbeds.
%
While \sysname cannot avoid this, we argue that our approach can
reduce validation effort.
%
Instead of validating each combination of simulators, components can
be validated individually and then composed.
%
This enables users to combine previously validated component
configurations into a full system.
%
We propose a public repository of validated component simulator
configurations to simplify re-use.
%
To ensure validity of these configurations over time, we imagine a
continuous-integration system, periodically re-running configurations and
recording the results.


\paragraph{Beyond networking.}
While we evaluate \sysname for network systems, our approach
generalizes beyond networking.
%
We have already demonstrated that our PCIe interface can support an
NVMe simulator.
%
Going forward, simulating PCIe attached accelerators, which are also
attracting growing interest in our community, should not require
changes to \sysname.
%
\sysname can also be easily extended with additional components or
interfaces, such as CXL~\cite{spec:cxl}.
%
We expect the emergence of further use-cases as architecture and
systems researchers continue to investigate specialized hardware.


\paragraph{Evaluating ASIC designs.}
Finally, we see evaluation of systems that include new ASIC components
as a driving use-case in the future.
%
While small ASIC designs with lower clock rates can often be evaluated
in physical testbeds with FPGAs, this is not possible for larger
designs or designs with fast clock speeds.
%
\sysname, on the other hand, can simulate ASIC RTL with arbitrary
frequencies, although FPGA accelerated RTL
simulations~\cite{karandikar:firesim} may be required for manageable
simulation times.
