\noindent
\emph{Appendices are supporting material that has not been
peer-reviewed.}

\section{Appendix}

\subsection{Modular Simulation Orchestration}
\label{ssec:appendix:orchestration}
Finally, an operational challenge arises for running simulations with \sysname.
Because we design \sysname without any centralized control, a simulation
consists entirely of interconnected component simulators.
Thus, to run a complete end-to-end simulation, a user has to start each
individual component simulator, while providing unique paths for the Unix
sockets and shared memory regions for each channel.
While this is manageable with very small simulations, the complexity rapidly
grows with simulation size, along with the additional challenges of cleanup,
collecting simulation logs, and monitoring for crashes.
An additional challenge, especially when running multiple simulations in
parallel, is that performance drastically degrades when overcommitting cores or
memory.
\sysname addresses both challenges with an orchestration framework for
assembling, running, and, if necessary, scheduling simulations.

\begin{figure}[h]%
\begin{lstlisting}[language=Python]
from simbricks import *
for rate in [10, 100, 200, 500, 1000]:
  e = Experiment('udp-' + str(rate))
  net = SwitchBM(e)

  s = Gem5Host(e, 'server')
  s.nic = I40eNIC(e)
  s.node_config = I40eLinuxNode()
  s.node_config.ip = '10.0.0.1'
  s.node_config.app = IperfUDPServer()

  c = Gem5Host(e, 'client')
  c.nic = I40eNIC(e)
  c.node_config = I40eLinuxNode()
  c.node_config.ip = '10.0.0.2'
  c.node_config.app = IperfUDPClient()
  c.node_config.app.server = '10.0.0.1'
  c.node_config.app.rate = rate

  experiments.append(e)
\end{lstlisting}\vspace{0.5em}%
\caption{An example of a simulation configuration in the \sysname
  orchestration framework.}%
\label{fig:orchestration}%
\Description{This figure shows pseudo-code but the text is included as
  in the PDF and as such should be accessible.}%
\end{figure}

Similar to other simulators with modular configuration we also implement our
orchestration in a scripting language.
The \sysname orchestration framework is designed as a collection of python
modules, and simulation experiments can be assembled by relying on arbitrary
python features.
In addition to the previously mentioned tasks, we also integrate functionality
to automatically generate customized disk images for host simulators, \eg with
different IP address configurations or to run applications with separate
parameters in individual hosts.
In \autoref{fig:orchestration} we show an example script.


\subsection{Inter-Simulator Message Transport}
\label{sec:appendix:shm}

\begin{figure}%
\begin{algorithmic}[0]%
  \State rxQueue, rxLen $\gets$ \Call{MapQueue}{rx}
  \State rxHead $\gets 0$
  \State txQueue, txLen $\gets$ \Call{MapQueue}{tx}
  \State txTail $\gets 0$
  \State
  \Procedure{PollMsg}{}
    \State msg $\gets$ \&rxQueue[rxHead]
    \While{msg->owner $\ne$ \texttt{CONSUMER}}
      \State \Call{Spin}{}
    \EndWhile
    \State \Call{ReadMemoryBarrier}{}
    \State rxHead $\gets$ (rxHead + 1) \% rxLen
    \State \Return{msg}
  \EndProcedure
  \Procedure{ReleaseMsg}{msg}
    \State msg->owner $\gets$ \texttt{PRODUCER}
  \EndProcedure
  \Procedure{AllocMsg}{}
    \State msg $\gets$ \&txQueue[txTail]
    \While{msg->owner $\ne$ \texttt{PRODUCER}}
      \State \Call{Spin}{}
    \EndWhile
    \State txTail $\gets$ (txTail + 1) \% txLen
    \State \Return{msg}
  \EndProcedure
  \Procedure{EnqueueMsg}{msg}
    \State \Call{WriteMemoryBarrier}{}
    \State msg->owner $\gets$ \texttt{CONSUMER}
  \EndProcedure
\end{algorithmic}%
\caption{\sysname multi-core shared memory message passing queue.
  \textsc{ReadMemoryBarrier} and \textsc{WriteMemoryBarrier} are compiler
  barriers to prevent re-ordering during optimization.}%
\label{fig:shm-queue}%
\Description{This figure shows pseudo-code but the text is included as
  in the PDF and as such should be accessible.}%
\end{figure}

\autoref{fig:shm-queue} shows pseudocode for the \sysname queue implementation.
To enable zero-copy implementation in simulators producer and consumer each have
separate functions for getting access to an available queue slot,
\textsc{PollMsg} for the consumer and \textsc{AllocMsg} for the producer, and
then releasing in when processing is complete, \textsc{ReleaseMsg} for the
consumer and \textsc{EnqueueMsg} for the producer.
The consumer uses its local head pointer to determine the slot the next message
is or will be in and then checks the type and ownership byte, re-trying if the
slot is marked by as owned by the producer.
After the consumer completes processing a message it marks the message as owned
by the consumer.
Symmetrically, the producer uses its local tail pointer to determine the slot
for the next message, if necessary waits until the slot is marked as
producer-owned, and resets the ownership bit to consumer after it places the
message in the slot.
Compiler memory barriers are necessary to prevent the compiler from reordering
memory accesses across accesses to the ownership bit, but with the strong X86
memory model no CPU memory barriers are necessary.

\subsubsection{Coherence Behavior}
To understand the performance properties, consider three key cases,
the queue is empty, the queue is full, and the queue is neither empty
nor full.
%
When the queue is empty, the consumer will spin on the last cache
line, which will be in the local L1 after the first access, and only
incurs an additional when the producer updates that cache line.
%
When the queue is full, the producer similarly waits for the next slot
to free up with the same coherence behavior.
%
Finally, when neither is the case, the consumer immediately finds a
message when polling and incurs a necessary miss that will fetch the
message.
Further, the CPU hardware prefetcher will likely already fetch the
next message as they are laid out sequentially in memory, thereby
avoiding a demand miss (but of course incurring the same coherence
traffic).
The producer does have to read the ownership flag incurring a miss,
but also immediately finds the empty slot, and the same prefetcher
behavior applies.

\subsection{\sysname Implementation Effort}
\label{ssec:apppendix:implcode}
\begin{table}%
  \centering%
  \begin{tabular}{ c c c }
      \toprule
      & \textbf{\sysname Component} & \textbf{Lines} \\
      \midrule
      \multirow{4}{50pt}{\sysname\newline core}
      & Message transport library & 1411 \\
      & NIC behavioral model library & 715 \\
      & Distributed simulation proxy & 2080 \\
      & Runtime orchestration & 2102 \\
      \midrule
      \multirow{2}{50pt}{Host simulators}
      & gem5 integration & 1265 \\
      & QEMU integration & 676 \\
      \midrule
      \multirow{4}{50pt}{NIC simulators}
      & Corundum Verilator & 1315 \\
      & Intel i40e model & 2900 \\
      & Corundum model & 911 \\
      & gem5 e1000 model & 2952 \\
      \midrule
      \multirow{5}{50pt}{Network simulators}
      & ns-3 integration & 158 \\
      & OMNeT++ integration & 208 \\
      & Tofino simulator integration & 330 \\
      & Ethernet switch model & 399 \\
      & Menshen RMT Verilator & 391 \\
      & Packet generator & 415 \\
      \midrule
      \multirow{1}{50pt}{Dev sims.}
      & FEMU SSD integration & 1005 \\
      \bottomrule\\
  \end{tabular}%
  \caption{Lines of code for the various components in \sysname, excluding
  blank lines and comments. For integrated simulators we only count
  adapter code.}%
  \label{tab:impl}%
\end{table}

\autoref{tab:impl} shows a per-component breakdown of the implementation effort
for \sysname, listing the number of lines of code.

\subsection{Performance for \sysname Configurations}
\label{ssec:appendix:simcombos}
\begin{table}%
\centering%
\begin{tabular}{lllrrrc}%
  \toprule
    \multicolumn{3}{c}{Simulators} & & &
    \multicolumn{1}{c}{Sim.}
    \\
    Host  & NIC & Net & T'put & Latency &
    \multicolumn{1}{c}{Time} &
    Det. \\
    \midrule
    QK & IB & SW & 4.37\,G & 71\,$\mu$s & 00:00:32 \\
  QK & IB & NS & 409\,M & 141\,$\mu$s & 00:00:32 \\
  QK & IB & TO & 1.92\,M & 6.6\,ms & 00:00:33 \\
  QK & CB & SW & 1.84\,G & 211\,$\mu$s & 00:00:29 \\
  QK & CB & NS & 429\,M & 294\,$\mu$s & 00:00:30 \\
  QK & CB & TO & 2.18\,M & 6.7\,ms & 00:00:33 \\
  QK & CV & SW & 81\,M & 3.4\,ms & 00:00:31 \\
  QK & CV & NS & 82\,M & 3.4\,ms & 00:00:32 \\
  QK & CV & TO & 2.31\,M & 23\,ms & 00:00:33 \\
\midrule
  QT & IB & SW & 8.85\,G & 17\,$\mu$s & 01:05:03 & (\checkmark) \\
  QT & IB & NS & 8.88\,G & 17\,$\mu$s & 01:06:43 & (\checkmark) \\
  QT & CB & SW & 3.74\,G & 28\,$\mu$s & 01:00:24 & (\checkmark) \\
  QT & CB & NS & 3.74\,G & 28\,$\mu$s & 00:59:41 & (\checkmark) \\
  QT & CV & SW & 6.55\,G & 32\,$\mu$s & 04:13:10 & (\checkmark) \\
  QT & CV & NS & 6.39\,G & 32\,$\mu$s & 04:13:13 & (\checkmark) \\
  G5 & IB & SW & 8.84\,G & 20\,$\mu$s & 12:51:41 & \checkmark \\
  G5 & IB & NS & 8.92\,G & 20\,$\mu$s & 12:49:46 & \checkmark \\
  G5 & CB & SW & 3.05\,G & 33\,$\mu$s & 09:20:48 & \checkmark \\
  G5 & CB & NS & 3.06\,G & 33\,$\mu$s & 09:26:13 & \checkmark \\
  G5 & CV & SW & 6.70\,G & 37\,$\mu$s & 10:23:26 & \checkmark \\
  G5 & CV & NS & 6.43\,G & 37\,$\mu$s & 10:21:28 & \checkmark %\\
 \\
    \bottomrule\\
\end{tabular}%
\caption{Performance for combinations of some of our component
    simulators.
    %
    Checkmarks mark deterministic combinations.
    %
    Host: QK is QEMU with KVM (functional simulation), QT
    is QEMU with timing, and G5 is gem5.
    %
    NIC: IB is the Intel behavioral model, CB the Corundum
    behavioral model, and CV the Corundum verilator model.
    %
    Network: SW is the switch behavioral model, NS is ns-3, TO is the
    Tofino model.
    }%
\label{table:crossproduct}%
\end{table}

\autoref{table:crossproduct} contains a cross-product of different simulators in
\sysname for host, NIC, and the network. This is an extended version
of \autoref{tab:modcombo} with the same experimental setup.
Note that with recent versions of QEMU we have found QEMU + timing
(QT) no longer to be fully deterministic and have instead observed
minor variations in simulation results.

\clearpage

\section{Artifact Appendix}

\subsection*{Abstract}

The \sysname artifact comprises two components, the source code of the
main simulator, and paper-specific parts (artifact scripts,
documentation, and data) to replicate the results in this
paper.

\subsection*{Scope}
Users interested in using \sysname in their work should refer to the
former, as this will continue to evolve over time, while the latter
remains stable (modulo bug fixes) to ensure reproducible results.

The artifact scripts can run all major and minor experiments in
the paper, except for the physical testbed baseline for the dctcp
experiments.
%
For deterministic simulations, results should be exactly reproducible.
%
Other measurements, especially simulation times, will vary based on
the hardware, but should be approximately reproducible on
similar hardware to what we describe.

\subsection*{Contents}
The artifact contains everything required to reproduce the results in
the paper: source code, instructions for building and running
\sysname, scripts for running experiments, and plotting scripts for
the graphs in the paper. We also include most of the execution logs we
generated for the experiments in this paper.

\subsection*{Hosting}
Both the main \sysname repo and the artifact package are hosted on
GitHub:
\begin{itemize}
  \item \textbf{Main \sysname source:} \newline
    \url{https://github.com/simbricks/simbricks}
  \item \textbf{Artifact package:} \newline
    \url{https://github.com/simbricks/sigcomm22-artifact}
\end{itemize}

For both we have tagged the version submitted for evaluation with
\texttt{sigcomm22-ae-submission}, and a stable version potentially
receiving bug-fixes will remain in the \texttt{sigcomm22-ae} branch.
The \texttt{main} branch will evolve and might contain breaking
changes.

We have also built docker specifically for the artifact that we link to in the
artifact README file.

\subsection*{Requirements}
The precise hardware requirements for each experiment vary
significantly and are detailed in the artifact repository. All
non-distributed experiments only require a single machine, but require
sufficient processor cores (varies per experiment up to 44). The
largest experiments also require around 192\,GB of RAM.

We have tested \sysname on Linux. The specific software dependencies
are provided by the documentation in the artifact repo.
