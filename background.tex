\section{Simulation Background}
Simulators employ techniques such as discrete event
simulation, binary translation, and hardware virtualization, to
simulate system components at various scales and levels of detail.
%
Network simulators, such as ns-2~\cite{software:ns2},
ns-3~\cite{software:ns3}, and OMNeT++~\cite{varga:omnetpp}, use
discrete event simulation to model packets traversing network
topologies.
%
Computer architecture simulators, such as gem5~\cite{binkert:gem5},
QEMU~\cite{software:qemu}, and Simics~\cite{magnusson:simics},
simulate full computer systems capable of running unmodified guest
software, including operating systems, with different and sometimes
configurable degrees of detail.
%
These simulators also include I/O devices, but often only implement
the minimum features for basic functionality.
%
Hardware RTL simulations, such as xsim~\cite{software:vivado_sim} and
Verilator~\cite{software:verilator}, help test and debug hardware
designs cycle by cycle against testbenches.
%
In all three cases
\emph{individual components are simulated in isolation}.

\paragraph{Advantages.}
The main motivation for simulation is that a physical implementation is often
not feasible.
%
Simulations are also \emph{portable}
as they decouple the simulated system from the host system.
%
Many are deterministic (with explicit seeds for randomness), providing
\emph{reproducible results}.
%
Simulators are also \emph{flexible}; implemented as software they can
be modified, and frequently offer parameters representing a broad
range of configurations.
%
Finally, simulations provide great \emph{visibility}, and can log
details about the system, without affecting behavior.

\paragraph{Disadvantages.}
Simulations also have some common drawbacks.
%
\emph{Long simulation times} are common -- architectural simulators
often only simulate hundreds or thousands of system cycles a
second~\cite{sutherland:nebula,karandikar:firesim}, and simulating a
few milliseconds of a large scale topology in ns-3 can take many
hours.
%
Different simulators strike different trade-offs between accuracy and
simulation time, depending on the intended use-case.
%
Finally, simulation results are only as good as the simulator, and may
not be representative unless \emph{validated} against a physical
testbed.

\paragraph{Comparison to Emulation.}
Emulations replicate externally visible behavior of a system without
modeling internal details, and typically run at close to interactive
speeds.
%
For example, Mininet~\cite{lantz:mininet} emulates OpenFlow networks with
multiple end-hosts running real Linux applications at near native speed on a
single physical host, by using Linux containers and kernel network features.
%
However, as emulation uses wall-clock time, it only works as long as
all components can keep up in real time.
%
Simulations, in contrast, rely on virtual time which can slow down without
affecting simulated behavior.
%
Additionally, emulation does not model internals of a system that could affect
system behavior, e.g., interactions between NIC and drivers.
%
As such, emulation is primarily useful for interactive testing or performance
evaluation when fidelity is not crucial.




\section{Systems Research Challenges}
\label{ssec:bg:sysresearch}
%
Systems research faces additional challenges that complicate using
simulation during prototyping and evaluation.

\begin{figure}%
\centering%
\begin{tikzpicture}[gnuplot]
%% generated with GNUPLOT 5.4p2 (Lua 5.4; terminal rev. Jun 2020, script rev. 114)
%% Tue 05 Jul 2022 09:49:52 PM CEST
\tikzset{every node/.append style={font={\fontsize{8.0pt}{9.6pt}\selectfont}}}
\path (0.000,0.000) rectangle (8.458,4.216);
\gpcolor{color=gp lt color border}
\gpsetlinetype{gp lt border}
\gpsetdashtype{gp dt solid}
\gpsetlinewidth{1.00}
\draw[gp path] (0.907,0.688)--(1.087,0.688);
\draw[gp path] (8.016,0.688)--(7.836,0.688);
\node[gp node right] at (0.907,0.688) {$0$};
\draw[gp path] (0.907,1.285)--(1.087,1.285);
\draw[gp path] (8.016,1.285)--(7.836,1.285);
\node[gp node right] at (0.907,1.285) {$2$};
\draw[gp path] (0.907,1.881)--(1.087,1.881);
\draw[gp path] (8.016,1.881)--(7.836,1.881);
\node[gp node right] at (0.907,1.881) {$4$};
\draw[gp path] (0.907,2.478)--(1.087,2.478);
\draw[gp path] (8.016,2.478)--(7.836,2.478);
\node[gp node right] at (0.907,2.478) {$6$};
\draw[gp path] (0.907,3.074)--(1.087,3.074);
\draw[gp path] (8.016,3.074)--(7.836,3.074);
\node[gp node right] at (0.907,3.074) {$8$};
\draw[gp path] (0.907,3.671)--(1.087,3.671);
\draw[gp path] (8.016,3.671)--(7.836,3.671);
\node[gp node right] at (0.907,3.671) {$10$};
\draw[gp path] (0.907,0.688)--(0.907,0.868);
\draw[gp path] (0.907,3.969)--(0.907,3.789);
\node[gp node center] at (0.907,0.515) {$0$};
\draw[gp path] (1.923,0.688)--(1.923,0.868);
\draw[gp path] (1.923,3.969)--(1.923,3.789);
\node[gp node center] at (1.923,0.515) {$20$};
\draw[gp path] (2.938,0.688)--(2.938,0.868);
\draw[gp path] (2.938,3.969)--(2.938,3.789);
\node[gp node center] at (2.938,0.515) {$40$};
\draw[gp path] (3.954,0.688)--(3.954,0.868);
\draw[gp path] (3.954,3.969)--(3.954,3.789);
\node[gp node center] at (3.954,0.515) {$60$};
\draw[gp path] (4.969,0.688)--(4.969,0.868);
\draw[gp path] (4.969,3.969)--(4.969,3.789);
\node[gp node center] at (4.969,0.515) {$80$};
\draw[gp path] (5.985,0.688)--(5.985,0.868);
\draw[gp path] (5.985,3.969)--(5.985,3.789);
\node[gp node center] at (5.985,0.515) {$100$};
\draw[gp path] (7.000,0.688)--(7.000,0.868);
\draw[gp path] (7.000,3.969)--(7.000,3.789);
\node[gp node center] at (7.000,0.515) {$120$};
\draw[gp path] (8.016,0.688)--(8.016,0.868);
\draw[gp path] (8.016,3.969)--(8.016,3.789);
\node[gp node center] at (8.016,0.515) {$140$};
\draw[gp path] (0.907,3.969)--(0.907,0.688)--(8.016,0.688)--(8.016,3.969)--cycle;
\node[gp node center,rotate=-270] at (0.232,2.328) {Throughput [Gbps]};
\node[gp node center] at (4.461,0.171) {Marking Threshold K [1500B]};
\node[gp node right] at (6.807,1.605) {ns-3};
\gpcolor{rgb color={0.580,0.000,0.827}}
\draw[gp path] (6.954,1.605)--(7.722,1.605);
\draw[gp path] (0.907,1.315)--(1.470,3.670)--(2.034,3.671)--(2.597,3.671)--(3.160,3.671)%
  --(3.724,3.671)--(4.287,3.671)--(4.850,3.671)--(5.413,3.671)--(5.977,3.671)--(6.540,3.671)%
  --(7.103,3.671)--(7.667,3.671);
\gpsetpointsize{4.00}
\gp3point{gp mark 1}{}{(0.907,1.315)}
\gp3point{gp mark 1}{}{(1.470,3.670)}
\gp3point{gp mark 1}{}{(2.034,3.671)}
\gp3point{gp mark 1}{}{(2.597,3.671)}
\gp3point{gp mark 1}{}{(3.160,3.671)}
\gp3point{gp mark 1}{}{(3.724,3.671)}
\gp3point{gp mark 1}{}{(4.287,3.671)}
\gp3point{gp mark 1}{}{(4.850,3.671)}
\gp3point{gp mark 1}{}{(5.413,3.671)}
\gp3point{gp mark 1}{}{(5.977,3.671)}
\gp3point{gp mark 1}{}{(6.540,3.671)}
\gp3point{gp mark 1}{}{(7.103,3.671)}
\gp3point{gp mark 1}{}{(7.667,3.671)}
\gp3point{gp mark 1}{}{(7.338,1.605)}
\gpcolor{color=gp lt color border}
\node[gp node right] at (6.807,1.310) {Physical Testbed (ground truth)};
\gpcolor{rgb color={0.000,0.620,0.451}}
\gpsetlinewidth{5.00}
\draw[gp path] (6.954,1.310)--(7.722,1.310);
\draw[gp path] (0.907,1.756)--(1.470,2.427)--(2.034,2.997)--(2.597,3.122)--(3.160,3.265)%
  --(3.724,3.381)--(4.287,3.459)--(4.850,3.519)--(5.413,3.537)--(5.977,3.557)--(6.540,3.566)%
  --(7.103,3.596)--(7.667,3.587);
\gp3point{gp mark 2}{}{(0.907,1.756)}
\gp3point{gp mark 2}{}{(1.470,2.427)}
\gp3point{gp mark 2}{}{(2.034,2.997)}
\gp3point{gp mark 2}{}{(2.597,3.122)}
\gp3point{gp mark 2}{}{(3.160,3.265)}
\gp3point{gp mark 2}{}{(3.724,3.381)}
\gp3point{gp mark 2}{}{(4.287,3.459)}
\gp3point{gp mark 2}{}{(4.850,3.519)}
\gp3point{gp mark 2}{}{(5.413,3.537)}
\gp3point{gp mark 2}{}{(5.977,3.557)}
\gp3point{gp mark 2}{}{(6.540,3.566)}
\gp3point{gp mark 2}{}{(7.103,3.596)}
\gp3point{gp mark 2}{}{(7.667,3.587)}
\gp3point{gp mark 2}{}{(7.338,1.310)}
\gpcolor{color=gp lt color border}
\node[gp node right] at (6.807,1.015) {\sysname (gem5 + i40e + ns-3)};
\gpcolor{rgb color={0.337,0.706,0.914}}
\gpsetlinewidth{1.00}
\draw[gp path] (6.954,1.015)--(7.722,1.015);
\draw[gp path] (0.907,1.708)--(1.470,2.737)--(2.034,2.985)--(2.597,3.122)--(3.160,3.205)%
  --(3.724,3.262)--(4.287,3.310)--(4.850,3.361)--(5.413,3.393)--(5.977,3.420)--(6.540,3.429)%
  --(7.103,3.465)--(7.667,3.462);
\gp3point{gp mark 3}{}{(0.907,1.708)}
\gp3point{gp mark 3}{}{(1.470,2.737)}
\gp3point{gp mark 3}{}{(2.034,2.985)}
\gp3point{gp mark 3}{}{(2.597,3.122)}
\gp3point{gp mark 3}{}{(3.160,3.205)}
\gp3point{gp mark 3}{}{(3.724,3.262)}
\gp3point{gp mark 3}{}{(4.287,3.310)}
\gp3point{gp mark 3}{}{(4.850,3.361)}
\gp3point{gp mark 3}{}{(5.413,3.393)}
\gp3point{gp mark 3}{}{(5.977,3.420)}
\gp3point{gp mark 3}{}{(6.540,3.429)}
\gp3point{gp mark 3}{}{(7.103,3.465)}
\gp3point{gp mark 3}{}{(7.667,3.462)}
\gp3point{gp mark 3}{}{(7.338,1.015)}
\gpcolor{color=gp lt color border}
\draw[gp path] (0.907,3.969)--(0.907,0.688)--(8.016,0.688)--(8.016,3.969)--cycle;
%% coordinates of the plot area
\gpdefrectangularnode{gp plot 1}{\pgfpoint{0.907cm}{0.688cm}}{\pgfpoint{8.016cm}{3.969cm}}
\end{tikzpicture}
%% gnuplot variables
%
\caption{Throughput for two dctcp flows in ns-3, a physical
      testbed, and a \sysname end-to-end simulation.}%
\label{fig:dctcp}%
\Description{Line graph showing the throughput on the y-axis and the
  dctcp marking threshold K for three configurations: a physical
  testbed (the ground thruth), a conventional network-only ns-3
  simulation, and \sysname (with gem5, i40e, and ns-3). The lines for
  the physical testbed and \sysnets match closely with just minor
  discrepancies. ns-3 on the other hand immediately reaches line-rate
  at K=10, where the physical testbed still only manages 6\,Gbps.}
\end{figure}

\paragraph{Not end-to-end.}
First and foremost, \emph{no existing simulator covers all
required components for network systems with sufficient features and
detail}, precluding end-to-end evaluation.
%
While existing simulators cover individual components, such as computer
architecture, hardware devices, and networks, they only do so in isolation with
no mechanism for combining them into complete systems.
%
As a result, we are left with non-end-to-end ``piecemeal'' evaluation,
where different components are evaluated in isolation~\cite{
  arashloo:tonic,mittal:irn,handley:ndp}.

We illustrate the pitfalls of piecemeal evaluation by comparing
dctcp~\cite{alizadeh:dctcp} congestion control behavior
in the ns-3 network simulator to a physical testbed.
%
As network speed increases and bottlenecks move to end-hosts,
congestion control incurs small variations in timing in the host
hardware and software which can affect
behavior~\cite{alizadeh:dctcp,mittal:timely,kumar:swift}.
%
However, ns-3 only models network and protocol behavior, and as
a result, does not capture these factors.
%
We set up two clients and two servers sharing a single 10G
bottleneck link with a 4000B MTU, and one large TCP flow
generated by iperf for each client-server pair.
%
\autoref{fig:dctcp} shows the throughput for varying dctcp
marking thresholds $K$.
%
The marking threshold balances queuing latency and throughput; a lower
threshold reduces queue length but risks under-utilizing links.
%
ns-3 underestimates the necessary threshold~\cite{alizadeh:dctcp} to
achieve line rate, as it does not model host processing variations,
particularly processing delay caused by OS interrupt scheduling.
%
Only an end-to-end evaluation of the full system captures such
intricacies.

\paragraph{Not scalable.}
Network and distributed systems frequently require evaluation on clusters beyond
tens of hosts to demonstrate scalability.
%
But for most simulators, already long simulation times increase super-linearly
with the size of the simulated system, making simulation of a large network
system an infeasible task.

\paragraph{Not modular.}
Using simulators for systems research often requires extending
existing simulators with additional functionality, \eg, adding a new
NIC to an architecture simulator.
%
These extensions are tied to a particular simulator, as different
simulators lack common internal interfaces,
%
This complicates apples-to-apples comparisons for future work that may use a
different simulator, \eg, to simulate a host with a different NIC, forcing the
same simulator to be used throughout the project cycle.
%
Finally, this tight integration complicates the implementation and releasing of
such extensions, as they often require maintaining a fork of the full simulator.
